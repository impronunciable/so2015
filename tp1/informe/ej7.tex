Este ejercicio parte de la observación del comportamiento de un scheduler llamado SchedMistery, del cual poseemos solamente una versión compilada y no su código fuente. Se plantea entonces entender su comportamiento y generar un scheduler SchedNoMistery que lo imite.

\bigskip

\textbf{Descripción del comportamiento observado}

Las observaciones de múltiples lotes, que serán detallados más adelante en la sección, nos llevó a entender las decisiones tomadas por el SchedMistery.

Este scheduler toma como parámetros 1 o más enteros. La cantidad de parámetros indica la cantidad de colas de prioridad para los procesos y cada parámetro
referencia el quantum de los procesos en cada cola de prioridad.

Nos encontramos frente a un round-robin de múltiples colas con prioridad. La forma en que cambian de prioridad los procesos se da en los siguientes casos:

- Cuando una tarea se desbloquea entonces sube un nivel de prioridad.
- Cuando una tarea pierde su quantum baja un nivel de prioridad.


\textbf{Implementación}

El comportamiento del scheduler queda descripto en el apartado anterior con lo cual nos vamos a limitar a explicitar las estructuras de datos intervinientes.

Para representar las múltiples colas utilizamos un vector de colas, cuya cantidad de elementos se corresponde con la cantidad de argumentos recibidos por el scheduler.
A la vez, un vector de enteros guarda un entero por cada cola de prioridad, determinando el quantum de los procesos que la contienen.
Resultó cómodo utilizar un mapa de entero a entero que, para cada pid devuelve la prioridad del proceso correspondiente.
Pensamos inicialmente en usar un struct para guardar información del proceso pero el mapa fue más efectivo y requiere menos cuidado a la hora de modificarse.

\bigskip

\textbf{Lotes de tareas de experimentación}

WIP

-El lote ej-7-1 (ej7/misterio-1.png) nos muestra que se hace un round robin y que las tareas desbloqueadas `ganan` prioridad

-El lote ej-7-2 (ej7/misterio-2.png) nos muestra que hay 1 cola de prioridad por cada parametro

-El lote ej-7-3 (ej7/misterio-3.png) nos muestra que al empezar una tarea tiene mayor prioridad que si ya estaba corriendo
