La implementación del servidor backend es análoga a la utilizada en su versión monoteísta, con la diferencia que se utiliza un thread distinto para cada jugador, y se utiliza la clase RWLock para sincronizar las lecturas y escrituras en el tablero. Utilizamos el llamado a ''pthread$\_$create(thread, NULL, atendedor$\_$de$\_$jugador, (void*) socketfd$\_$cliente)'', para obtener un nuevo thread correspondiente a un cliente. La función $atendedor\_de\_jugador$, es similar a la correspondiente a su versión monoteísta, pero con las consideraciones necesarias para la sincronización en el tablero. En este sentido utilizamos un RWLock global por cada posición del tablero, específicamente tenemos un $ vector<vector<RWLock> > tablero\_letras\_rwlocks$ y un $vector<vector<RWLock> > tablero\_palabras\_rwlocks $. Cada vez que escribimos en alguna posición bloque utilizamos el RWLock correspondiente, y llamamos a $ wlock() $, y $ wunlock() $, y cada vez que leemos alguna posición, utilizamos nuevamente el RWLock correspondiente, pero llamando a $ rlock() $ y $ runlock() $.

Sin embargo en ocasiones hay que tomar el lock de lectura o de escritura de varias posiciones antes de efectuar la operación, ya que la misma involucra muchas posiciones en conjunto. Cuando el comando recibido por el backend es MSG_PALABRA, se debe almacenar en el casillero la palabra completa. En el siguiente fragmento encontramos el pseudicódigo utilizado para almacenar a la misma. 

 
\begin{lstlisting}
for (list<Casillero>::const_iterator casillero = palabra_actual.begin(); casillero != palabra_actual.end(); casillero++) {
                tablero_palabras_rwlocks[casillero->fila][casillero->columna].wlock();
            }
            // las letras acumuladas conforman una palabra completa, escribirlas en el tablero de palabras y borrar las letras temporales
            for (list<Casillero>::const_iterator casillero = palabra_actual.begin(); casillero != palabra_actual.end(); casillero++) {
                tablero_palabras[casillero->fila][casillero->columna] = casillero->letra;
            }
             for (list<Casillero>::const_iterator casillero = palabra_actual.begin(); casillero != palabra_actual.end(); casillero++) {
                tablero_palabras_rwlocks[casillero->fila][casillero->columna].wunlock();
            }
\end{lstlisting}

En caso de que se asigne erróneamente por parte del usuario una letra, toda la palabra actualmente almacenada debe limpiarse, para eso se llama a la función $ quitar\_letras $, la cual primero toma el lock de todas las letras actualmente almacenadas, y luego las setea en VACIO, y finalmente libera el lock de todas ellas.
En el siguiente fragmento encontramos el pseudocódigo de la misma. 

\begin{lstlisting}
void quitar_letras(list<Casillero>& palabra_actual) {
    for (list<Casillero>::const_iterator casillero = palabra_actual.begin(); casillero != palabra_actual.end(); casillero++) {
        tablero_letras_rwlocks[casillero->fila][casillero->columna].wlock();
    }
    for (list<Casillero>::const_iterator casillero = palabra_actual.begin(); casillero != palabra_actual.end(); casillero++) {
        tablero_letras[casillero->fila][casillero->columna] = VACIO;
    }
    for (list<Casillero>::const_iterator casillero = palabra_actual.begin(); casillero != palabra_actual.end(); casillero++) {
        tablero_letras_rwlocks[casillero->fila][casillero->columna].wunlock();
    }
    palabra_actual.clear();
}
\end{lstlisting}

Cuando se envía un tablero, hay que tomar el lock de lectura de todas las posiciones, enviar el tablero y luego liberarlo. En el siguiente fragmento encontramos el pseudocódigo.

\begin{lstlisting}
int enviar_tablero(int socket_fd) {
    char buf[MENSAJE_MAXIMO+1];
    sprintf(buf, "STATUS ");
    int pos = 7;
    //tomamos el rlock de todas las posiciones para que no se pueda escribir
    for (unsigned int fila = 0; fila < alto; ++fila) {
        for (unsigned int col = 0; col < ancho; ++col) {
            tablero_palabras_rwlocks[fila][col].rlock();
        }
    }
    for (unsigned int fila = 0; fila < alto; ++fila) {
        for (unsigned int col = 0; col < ancho; ++col) {
            char letra = tablero_palabras[fila][col];
            buf[pos] = (letra == VACIO)? '-' : letra;
            pos++;
        }
    }
    buf[pos] = 0; //end of buffer
    for (unsigned int fila = 0; fila < alto; ++fila) {
        for (unsigned int col = 0; col < ancho; ++col) {
            tablero_palabras_rwlocks[fila][col].runlock();
        }
    }

    return enviar(socket_fd, buf);
}
\end{lstlisting}

Para la función $ es\_ficha\_valida\_en\_palabra $, al principio de la misma tomamos el lock de la palabra almacenada hasta el momento, y de la nueva ficha, para ambos el lock de lectura. Nos fijamos si es válida, y antes de retornar, liberamos el los lock tomados. En el siguiente fragmento encontramos el pseudocódigo utilizado para tomar el lock:

\begin{lstlisting}

tablero_palabras_rwlocks[ficha.fila][ficha.columna].rlock();

   for (list<Casillero>::const_iterator casillero = palabra_actual.begin(); casillero != palabra_actual.end(); casillero++) {
        tablero_letras_rwlocks[casillero->fila][casillero->columna].rlock();
    }

\end{lstlisting}


y para liberarlo:

\begin{lstlisting}

tablero_palabras_rwlocks[ficha.fila][ficha.columna].runlock();
        for (list<Casillero>::const_iterator casillero = palabra_actual.begin(); casillero != palabra_actual.end(); casillero++) {
            tablero_letras_rwlocks[casillero->fila][casillero->columna].runlock();
        }
        return false;
   \end{lstlisting}

