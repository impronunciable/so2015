Implementamos la clase $RWLock$, la cual nos permite administrar el uso concurrente de algún recurso global por distintos threads. Con la misma podemos tener varios threads leyendo simultáneamente el valor del recurso, siempre que no haya alguno escribiendo y podemos tener solo uno escribiendo.

Para la misma utilizamos los atributos privados $ readers $, de tipo \textbf{int}, que indica la cantidad de threads que se encuentran actualmente leyendo el recurso, $ writing $, de tipo \textbf{bool}, para saber si hay algún thread leyendo, $ lock $ de tipo \textbf{pthread_cond_t}, que indica si hay algún escritor o lector utilizando el recurso, $ mutex $ de tipo \textbf{pthread_mutex_t}, asociado a la variable de condición $ lock $, $ turnstile $ de tipo \textbf{pthread_mutex_t}, utilizado para evitar la inanición. 

Se implementaron para la misma las funciones $ rlock() $, $ wlock() $, $ runlock() $, $ wunlock() $. En el siguiente fragmento podemos encontrar el pseudocódigo de las mismas:

\begin{lstlisting}
RWLock :: RWLock() {

    readers = 0;
    writing = false;
    pthread_cond_init(&lock, NULL);
    pthread_mutex_init(&mutex, NULL);
    pthread_mutex_init(&turnstile, NULL);
}

void RWLock :: rlock() {
    //pthread_rwlock_rdlock(&(this->rwlock));
    pthread_mutex_lock(&turnstile);
    pthread_mutex_unlock(&turnstile);
    pthread_mutex_lock(&mutex);
    //los lectores se quedan esperando que se termine de escribir
    while(writing){
        pthread_cond_wait(&lock, &mutex);
    }
    readers++;
    pthread_mutex_unlock(&mutex);
}

void RWLock :: wlock() {

    // Llega un escritor, no pueden pasar mas lectores
    pthread_mutex_lock(&turnstile);
    pthread_mutex_lock(&mutex);
    while(writing || readers > 0){
        pthread_cond_wait(&lock, &mutex);
    }
    writing = true;
    pthread_mutex_unlock(&mutex);
}

void RWLock :: runlock() {
    pthread_mutex_lock(&mutex);
    readers--;
    if (readers == 0){
        pthread_cond_signal(&lock);
    }
    pthread_mutex_unlock(&mutex);
}

void RWLock :: wunlock() {
    pthread_mutex_lock(&mutex);
    writing = false;
    pthread_cond_broadcast(&lock); // Broadcast para despertar tambien a readers
    // Desbloqueo el turnstile, para que pasen lectores/escritores
    pthread_mutex_unlock(&turnstile);
    pthread_mutex_unlock(&mutex);
}
\end{lstlisting}


La primera permite que todos los thread lectores del recurso se queden esperando hasta que el thread que esta escribiendo termine, cuando este último termina se llama a $ wunlock() $, la cual ''despierta'' a todos los los lectores que se encuentren esperando. Esto permite que las condiciones de haber algún lector o algún escritor sean mutuamente excluyentes, lo cual se mantendrá invariante a lo largo de la ejecución. La función $ wlock() $ permite que los threads escritores se queden esperando a que no haya lectores ni escritores, como son mutuamente excluyentes en realidad se queda esperando una sola de estas condiciones. La función $ runlock() $, en caso de que la nueva cantidad de lectores del recurso sea cero, ''despertara'' al escritor esperando que no hayan mas lectores. La función $ wunlock() $ ''despertara'' además de a los lectores, a los escritores, para esto utilizamos la función pthread_cond_broadcast(lock).

Sin embargo en este contexto es posible que un escritor se quede esperando indefinidamente, en caso de que siempre aparezcan nuevos lectores, y entonces nunca $ readers $ llegue a cero. Esto es un caso de inanición, para evitarla se utilizo la variable $ turnstile $ de tipo \textbf{pthread_mutex_unlock}. Esta permite que cuando un thread que desee escribir en el recurso, al llamar a $ wlock() $, se toma el mutex de $ turnstile $,(linea 1) entonces los lectores que aparecen después de este escritor deben esperar a que el escritor libere el mutex (linea 11). Cuando cada lector toma el mutex de $ turnstile $, lo libera enseguida (lineas 1 y 2 de $ rlock() $), entonces el mismo queda disponible para algún otro escritor o lector. Que thread exactamente tome el mutex, dependerá de de la política de scheduling. En esta situación también podría pasar que nunca se elija algún escritor, y siempre sigan apareciendo mas lectores. Sin embargo se espera que en algún momento el escritor sea elegido para tomar el mutex, ya sea por cierta política de scheduling específica, o debido a que siempre existen posibilidades de ser elegido, la probabilidad de nunca ser elegido es cero.
