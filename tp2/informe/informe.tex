\documentclass[a4paper]{article}
\usepackage[spanish]{babel}
\usepackage[utf8]{inputenc}
\usepackage{fancyhdr}
%\usepackage{charter}   % tipografía
\usepackage{graphicx}
\usepackage{makeidx}
\usepackage{mathptmx}

\usepackage{float}
\usepackage{amsmath, amsthm, amssymb}
\usepackage{amsfonts}
\usepackage{sectsty}
\usepackage{wrapfig}
\usepackage{listings} % necesario para el resaltado de sintaxis
\usepackage{caption}
%%%%%%%%%% Paquete para hacer grafos
%%% ver link http://www.texample.net/tikz/examples/bridges-of-konigsberg/
%\usepackage{fullpage}
%\usepackage{fourier}
\usepackage{tikz}
\usetikzlibrary{arrows,%
                shapes,positioning}
                
\thispagestyle{empty}
%%%%%%%%%% Fin paquete para hacer grafos
%%%%%%%%%%
\usepackage{hyperref} % agrega hipervínculos en cada entrada del índice
\hypersetup{          % (en el pdf)
    colorlinks=true,
    linktoc=all,
    citecolor=black,
    filecolor=black,
    linkcolor=black,
    urlcolor=black
}

\usepackage{color} % para snippets de código coloreados
\usepackage{fancybox}  % para el sbox de los snippets de código

\definecolor{litegrey}{gray}{0.94}

% \newenvironment{sidebar}{%
% 	\begin{Sbox}\begin{minipage}{.85\textwidth}}%
% 	{\end{minipage}\end{Sbox}%
% 		\begin{center}\setlength{\fboxsep}{6pt}%
% 		\shadowbox{\TheSbox}\end{center}}
% \newenvironment{warning}{%
% 	\begin{Sbox}\begin{minipage}{.85\textwidth}\sffamily\lite\small\RaggedRight}%
% 	{\end{minipage}\end{Sbox}%
% 		\begin{center}\setlength{\fboxsep}{6pt}%
% 		\colorbox{litegrey}{\TheSbox}\end{center}}

\newenvironment{codesnippet}{%
	\begin{Sbox}\begin{minipage}{\textwidth}\sffamily\small}%
	{\end{minipage}\end{Sbox}%
		\begin{center}%
		\colorbox{litegrey}{\TheSbox}\end{center}}



\usepackage{fancyhdr}
\pagestyle{fancy}

%\renewcommand{\chaptermark}[1]{\markboth{#1}{}}
\renewcommand{\sectionmark}[1]{\markright{\thesection\ - #1}}

\fancyhf{}

\fancyhead[LO]{Sección \rightmark} % \thesection\
\fancyfoot[LO]{\small{Jorge Porto, Dan Zajdband y Andreas Sturmer}}
\fancyfoot[RO]{\thepage}
\renewcommand{\headrulewidth}{0.5pt}
\renewcommand{\footrulewidth}{0.5pt}
\setlength{\hoffset}{-0.8in}
\setlength{\textwidth}{16cm}
%\setlength{\hoffset}{-1.1cm}
%\setlength{\textwidth}{16cm}
\setlength{\headsep}{0.5cm}
\setlength{\textheight}{25cm}
\setlength{\voffset}{-0.7in}
\setlength{\headwidth}{\textwidth}
\setlength{\headheight}{13.1pt}

\renewcommand{\baselinestretch}{1.1}  % line spacing


\usepackage{underscore}
\usepackage{caratula}
\usepackage{url}
\usepackage{color}
\usepackage{clrscode3e} % necesario para el pseudocodigo (estilo Cormen)

%\usepackage{algorithm}
%\usepackage{algorithmic}
\usepackage{algorithm}[1]
\usepackage{algorithmic}[1]
%\usepackage{algpseudocode}

\definecolor{dkgreen}{rgb}{0,0.6,0}
\definecolor{gray}{rgb}{0.5,0.5,0.5}
\definecolor{mauve}{rgb}{0.58,0,0.82}

\definecolor{gray}{gray}{0.5}
\definecolor{light-gray}{gray}{1}
\definecolor{orange}{rgb}{1,0.5,0}

\lstset{frame=tb,
  language=C++,
  aboveskip=3mm,
  belowskip=3mm,
  showstringspaces=false,
  columns=flexible,
  basicstyle={\small\ttfamily},
  keywordstyle=\color{blue},
  commentstyle=\color{gray},
  stringstyle=\color{mauve},
  breaklines=true,
  breakatwhitespace=true,
  tabsize=3,
  numbers=left,
  xleftmargin=2em,
  frame=single,
  framexleftmargin=2em,
  numbersep=5pt,                   % how far the line-numbers are from the code
  numberstyle=\small\color{gray} % the style that is used for the line-numbers
 }
 
 \lstdefinestyle{customc}{
  backgroundcolor=\color{light-gray},
  belowcaptionskip=1\baselineskip,
  breaklines=true,
  numbers=left,
  xleftmargin=\parindent,
  language=C++,
  showstringspaces=false,
  basicstyle=\footnotesize\ttfamily,
  keywordstyle=\bfseries\color{blue},
  commentstyle=\itshape\color{gray},
  identifierstyle=\color{black},
  stringstyle=\color{orange},
}

\begin{document}


\thispagestyle{empty}
\materia{Sistemas Operativos}
\submateria{Segundo Cuatrimestre de 2015}
\titulo{Trabajo Práctico 2: Pthreads}
\subtitulo{SOScrabel}
\integrante{}{}{} % por cada integrante (apellido, nombre) (n° libreta) (e-mail)
\integrante{}{}{}
\integrante{}{}{}

\maketitle
\newpage

\thispagestyle{empty}
\vfill

\thispagestyle{empty}
\vspace{1.5cm}
\tableofcontents
\newpage

%\normalsize

\newpage
\section{Introducción}
\setcounter{page}{1}
En este trabajo se implementará el servidor backend para el juego Scrabble de la empresa HaSObro del enunciado del trabajo practico. El mismo se comunica por TCP, con un servidor frontend, y permite la conexión simultánea de múltiples jugadores. Para evitar posibles conflictos entre los distintos jugadores al momento de leer o escribir una letra al mismo tiempo en el tablero, se implementará la clase RWLock.

%\newpage

%\newpage
\section{Read-Write Locks}
Implementamos la clase RWLock, la cual nos permite administrar el uso concurrente de algún recurso global por distintos threads. Con la misma podemos tener varios threads leyendo simultáneamente el valor del recurso, siempre que no halla alguno escribiendo, y podemos tener solo uno escribiendo.

Para la misma utilizamos los atributos privados $ readers $, de tipo \textbf{int}, que indica la cantidad de threads que se encuentran actualmente leyendo el recurso, $ writing $, de tipo \textbf{bool}, para saber si  hay algún thread leyendo, $ lock $ de tipo \textbf{pthread_cond_t}, que indica si hay algún escritor o lector utilizando el recurso, $ mutex $ de tipo \textbf{pthread_mutex_t}, asociado a la variable de condición $ lock $, $ turnstile $ de tipo \textbf{pthread_mutex_t}, utilizado para evitar la inanición. 

Se implementaron para la misma las funciones $ rlock() $, $ wlock() $, $ runlock() $, $ wunlock() $. En el siguiente fragmento podemos encontrar el pseudocódigo de las mismas:

\begin{lstlisting}
RWLock :: RWLock() {

    readers = 0;
    writing = false;
    pthread_cond_init(&lock, NULL);
    pthread_mutex_init(&mutex, NULL);
    pthread_mutex_init(&turnstile, NULL);
}

void RWLock :: rlock() {
    //pthread_rwlock_rdlock(&(this->rwlock));
    pthread_mutex_lock(&turnstile);
    pthread_mutex_unlock(&turnstile);
    pthread_mutex_lock(&mutex);
    //los lectores se quedan esperando que se termine de escribir
    while(writing){
        pthread_cond_wait(&lock, &mutex);
    }
    readers++;
    pthread_mutex_unlock(&mutex);
}

void RWLock :: wlock() {

    // Llega un escritor, no pueden pasar mas lectores
    pthread_mutex_lock(&turnstile);
    pthread_mutex_lock(&mutex);
    while(writing || readers > 0){
        pthread_cond_wait(&lock, &mutex);
    }
    writing = true;
    pthread_mutex_unlock(&mutex);
}

void RWLock :: runlock() {
    pthread_mutex_lock(&mutex);
    readers--;
    if (readers == 0){
        pthread_cond_signal(&lock);
    }
    pthread_mutex_unlock(&mutex);
}

void RWLock :: wunlock() {
    pthread_mutex_lock(&mutex);
    writing = false;
    pthread_cond_broadcast(&lock); // Broadcast para despertar tambien a readers
    // Desbloqueo el turnstile, para que pasen lectores/escritores
    pthread_mutex_unlock(&turnstile);
    pthread_mutex_unlock(&mutex);
}
\end{lstlisting}


La primera permite que todos los thread lectores del recurso se queden esperando hasta que el thread que esta escribiendo termine, cuando este último termina se llama a $ wunlock() $, la cual ''despierta'' a todos los los lectores que se encuentren esperando. Esto permite que las condiciones de haber algún lector o algún escritor sean mutuamente excluyentes, lo cual se mantendrá invariante a lo largo de la ejecución. La función $ wlock() $ permite que los thread escritores se queden esperando a que no halla lectores ni escritores, como son mutuamente excluyentes en realidad se queda esperando una sola de estas condiciones. La función $ runlock() $, en caso de que la nueva cantidad de lectores del recurso sea cero, ''despertara'' al escritor esperando que no halla mas lectores. La función $ wunlock() $ ''despertara'' además de a los lectores, a los escritores, para esto utilizamos la función pthread_cond_broadcast(lock).

Sin embargo en este contexto es posible que un escritor se quede esperando indefinidamente, en caso de que siempre aparezcan nuevos lectores, y entonces nunca $ readers $ llegue a cero. Esto es un caso de inanición, para evitarla se utilizo la variable $ turnstile $ de tipo \textbf{pthread_mutex_unlock}. 


\newpage
\section{Scrabble}
La implementación del servidor backend es análoga a la utilizada en su versión monoteísta, con la diferencia que se utiliza un thread distinto para cada jugador, y se utiliza la clase RWLock para sincronizar las lecturas y escrituras en el tablero. Utilizamos el llamado a ''pthread$\_$create(thread, NULL, atendedor$\_$de$\_$jugador, (void*) socketfd$\_$cliente)'', para obtener un nuevo thread correspondiente a un cliente. La función $atendedor\_de\_jugador$, es similar a la correspondiente a su versión monoteísta, pero con las consideraciones necesarias para la sincronización en el tablero. En este sentido utilizamos un RWLock global por cada posición del tablero, específicamente tenemos un $ vector<vector<RWLock> > tablero\_letras\_rwlocks$ y un $vector<vector<RWLock> > tablero\_palabras\_rwlocks $. Cada vez que escribimos en alguna posición bloque utilizamos el RWLock correspondiente, y llamamos a $ wlock() $, y $ wunlock() $, y cada vez que leemos alguna posición, utilizamos nuevamente el RWLock correspondiente, pero llamando a $ rlock() $ y $ runlock() $.

Sin embargo en ocasiones hay que tomar el lock de lectura o de escritura de varias posiciones antes de efectuar la operación, ya que la misma involucra muchas posiciones en conjunto. Cuando el comando recibido por el backend es MSG_PALABRA, se debe almacenar en el casillero la palabra completa. En el siguiente fragmento encontramos el pseudicódigo utilizado para almacenar a la misma. 

 
\begin{lstlisting}
for (list<Casillero>::const_iterator casillero = palabra_actual.begin(); casillero != palabra_actual.end(); casillero++) {
                tablero_palabras_rwlocks[casillero->fila][casillero->columna].wlock();
            }
            // las letras acumuladas conforman una palabra completa, escribirlas en el tablero de palabras y borrar las letras temporales
            for (list<Casillero>::const_iterator casillero = palabra_actual.begin(); casillero != palabra_actual.end(); casillero++) {
                tablero_palabras[casillero->fila][casillero->columna] = casillero->letra;
            }
             for (list<Casillero>::const_iterator casillero = palabra_actual.begin(); casillero != palabra_actual.end(); casillero++) {
                tablero_palabras_rwlocks[casillero->fila][casillero->columna].wunlock();
            }
\end{lstlisting}

En caso de que se asigne erróneamente por parte del usuario una letra, toda la palabra actualmente almacenada debe limpiarse, para eso se llama a la función $ quitar\_letras $, la cual primero toma el lock de todas las letras actualmente almacenadas, y luego las setea en VACIO, y finalmente libera el lock de todas ellas.
En el siguiente fragmento encontramos el pseudocódigo de la misma. 

\begin{lstlisting}
void quitar_letras(list<Casillero>& palabra_actual) {
    for (list<Casillero>::const_iterator casillero = palabra_actual.begin(); casillero != palabra_actual.end(); casillero++) {
        tablero_letras_rwlocks[casillero->fila][casillero->columna].wlock();
    }
    for (list<Casillero>::const_iterator casillero = palabra_actual.begin(); casillero != palabra_actual.end(); casillero++) {
        tablero_letras[casillero->fila][casillero->columna] = VACIO;
    }
    for (list<Casillero>::const_iterator casillero = palabra_actual.begin(); casillero != palabra_actual.end(); casillero++) {
        tablero_letras_rwlocks[casillero->fila][casillero->columna].wunlock();
    }
    palabra_actual.clear();
}
\end{lstlisting}

Cuando se envía un tablero, hay que tomar el lock de lectura de todas las posiciones, enviar el tablero y luego liberarlo. En el siguiente fragmento encontramos el pseudocódigo.

\begin{lstlisting}
int enviar_tablero(int socket_fd) {
    char buf[MENSAJE_MAXIMO+1];
    sprintf(buf, "STATUS ");
    int pos = 7;
    //tomamos el rlock de todas las posiciones para que no se pueda escribir
    for (unsigned int fila = 0; fila < alto; ++fila) {
        for (unsigned int col = 0; col < ancho; ++col) {
            tablero_palabras_rwlocks[fila][col].rlock();
        }
    }
    for (unsigned int fila = 0; fila < alto; ++fila) {
        for (unsigned int col = 0; col < ancho; ++col) {
            char letra = tablero_palabras[fila][col];
            buf[pos] = (letra == VACIO)? '-' : letra;
            pos++;
        }
    }
    buf[pos] = 0; //end of buffer
    for (unsigned int fila = 0; fila < alto; ++fila) {
        for (unsigned int col = 0; col < ancho; ++col) {
            tablero_palabras_rwlocks[fila][col].runlock();
        }
    }

    return enviar(socket_fd, buf);
}
\end{lstlisting}

Para la función $ es_ficha_valida_en_palabra $, al principio de la misma tomamos el lock de la palabra almacenada hasta el momento, y de la nueva ficha, nos fijamos si es válida, y antes de retornar, liberamos el los lock tomados.


%\input{apendice}



%\vspace*{0.5cm}

%\begin{lstlisting}
%int main(){
%  return 0;
%}
%\end{lstlisting}


%\vspace*{0.5cm}

%\newpage
%\subsection{Código del Problema 3}

%\begin{lstlisting}
%int main(){
%  return 0;
%}
%\end{lstlisting}

%\vspace*{0.5cm}

%\begin{lstlisting}
%int main(){
%  return 0;
%}
%\end{lstlisting}




\end{document}
